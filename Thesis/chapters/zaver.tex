\chapter{Závěr}
Cílem této bakalářské práce bylo vyvinout funkční prototyp mobilní aplikace pro zobrazení digitálních dvojčat stavebních projektů v rozšířené realitě. Aplikace měla umožnit veřejnosti prohlížet plánované stavby v reálném prostředí a~poskytovat zpětnou vazbu prostřednictvím hodnocení a~komentářů.

Navržený systém využívá architekturu klient-server a~skládá se ze tří hlavních komponent: mobilní aplikace implementované v prostředí Unity s využitím jazyka C\#, serverového REST API vytvořeného ve frameworku Flask a~webového rozhraní pro správu modelů digitálních dvojčat. Datová vrstva je realizována dokumentově orientovanou databází MongoDB, která umožňuje flexibilní ukládání metadat projektů, uživatelských hodnocení a~cachovaných mapových dlaždic.

Mobilní aplikace umožňuje uživatelům vyhledávat dostupné projekty pomocí interaktivního mapového rozhraní, zobrazovat 3D modely ve formátu GLB v prostředí rozšířené reality a~odesílat zpětnou vazbu formou hvězdičkového hodnocení a~textových komentářů. Webové rozhraní poskytuje kompletní správu projektů včetně nahrávání, úpravy a~mazání. Webové rozhraní také umožňuje procházet seznam jednotlivých projektů, kde je zobrazen jejich popis, průměrné hodnocení, název a~náhledový obrázek, a~zároveň poskytuje možnost zobrazení všech hodnocení a~komentářů.

Hlavním technickým přínosem práce je implementace vlastního mapového systému včetně odvození matematického vztahu pro transformaci geografických souřadnic do souřadnicového systému v Unity. Tento přístup byl nutný kvůli absenci vhodných bezplatných mapových knihoven pro Unity a~umožňuje přesné zobrazení projektů na mapě při různých úrovních přiblížení. Implementace cache systému pro mapové dlaždice s expiračním časem 10 dní výrazně snížila počet dotazů na Google Maps Static API a~zrychlila načítání mapy.

Srovnání s existujícími řešeními (Trimble Connect AR, Dalux BIM Viewer, Augment) ukazuje, že vytvořený prototyp obsazuje dosud nevyužitou kombinaci. Jednoduché, veřejně přístupné řešení zaměřené na hodnocení v územním plánování. Profesionální nástroje nabízejí pokročilé funkce a~vysokou přesnost, ale jsou finančně a~technicky náročné a~cílí na stavební firmy, nikoli na veřejnost.

Prototyp demonstruje technickou proveditelnost využití rozšířené reality pro vizualizaci digitálních dvojčat stavebních projektů. Budoucí vývoj by měl bezpochyby zahrnovat i autentizaci uživatelů v mobilní aplikaci, což by zamezilo nežádoucímu spamování komentářů a~hodnocení. Dále by mohla být implementována autentizace oprávněných uživatelů ve webové aplikaci pro lepší kontrolu nad správou projektů a integrace analytických nástrojů pro spracování zpětné vazby.