\chapter{Teoretická část}
\section{Digitální dvojčata a~územní plánování}
V dnešní době, kdy města čelí významným výzvám jako je rychlý populační růst, omezené zdroje, klimatické změny a~tlak na udržitelný rozvoj, je nezbytné využívat pokročilé nástroje pro územní plánování. Tradiční přístupy často nedostačují k řešení složitých a~dynamických problémů moderních měst. V~tomto kontextu se stále více prosazují technologie jako Internet věcí (IoT), umělá inteligence (AI) a~imerzivní technologie jako například virtuální a~rozšířená realita. \cite{marin2023}

\subsection{Digitální dvojče}
Digitální dvojče představuje procesy a~metody popisující a~modelující vlastnosti, chování, vznik a~fungování fyzických objektů prostřednictvím digitálních technologií. Model digitálního dvojčete plně odpovídá svému fyzickému protějšku v~reálném světě a~je schopen v~reálném čase simulovat jeho chování a~výkon. Digitální dvojče tak tvoří dynamický a~přesný digitální obraz fyzického objektu, který je průběžně aktualizován na základě dat z reálného prostředí. Tento model může umožňovat nejen monitorování, ale i predikci jeho chování a~optimalizaci provozu. Digitální dvojče se tak nemusí omezovat pouze na popis geometrie a~funkcí, ale může zahrnovat i celý životní cyklus fyzického objektu, včetně výstavby, údržby a~provozu.\cite{wang2020digital}

\subsection{Využití digitálních dvojčat v~územním plánování}
Digitální dvojčata umožňují simulaci různých urbanistických scénářů a~předpovídání dopadů plánovaných rozhodnutí, což vede k optimalizaci využití zdrojů, zvýšení udržitelnosti a~zlepšení kvality života obyvatel. Tato technologie podporuje datově podložené rozhodování a~umožňuje testování inovativních řešení v~bezpečném virtuálním prostředí, čímž minimalizuje náklady a~rizika reálných experimentů. Díky tomu se digitální dvojčata stávají klíčovým nástrojem moderního územního plánování a~správy měst.\cite{marin2023}

\pagebreak
\section{Rozšířená realita}
Rozšířená realita (\textit{Augmented Reality}) je technologie, která v~reálném čase spojuje fyzický svět s digitálními prvky, a~to ve formě 2D i 3D objektů, které jsou přirozeně začleněné do reálného prostředí. Umožňuje uživatelům interagovat s těmito digitálními objekty pomocí několika metod ovládání, jako jsou například pohyb zařízení, dotykové ovládání nebo hlasové příkazy. \cite{AROverview}

V českém jazyce se běžně používá termín \uv{rozšířená realita} jako překlad anglického \textit{Augmented Reality} (AR). Tento překlad však není zcela přesný. Pojem \textit{augmented} doslova znamená \uv{posílený} či \uv{rozšířený o něco navíc}, a~vhodnějším překladem by tak mohl být například \uv{posílená realita}. Nicméně, v~češtině se již ustálil výraz \uv{rozšířená realita}, a~tak jej budu užívat v~rámci této práce. \cite{BeyondAR}

Zároveň je třeba rozlišovat pojem \textit{Extended Reality} (XR), který označuje zastřešující termín pro technologie jako virtuální realita (\textit{Virtual Reality}, VR), smíšená realita (\textit{Mixed Reality}, MR) a~právě rozšířená realita (\textit{Augmented Reality}, AR). XR tedy pokrývá celé spektrum imerzivních technologií od posílené a~smíšené reality po čisté virtuální. \cite{BeyondAR}

Spektrum na následujícím obrázku přehledně ukazuje přechod mezi jednotlivými technologiemi.


\begin{figure}[H]
    \centering
    \includegraphics[width=\textwidth]{images/spectrum}
    \caption{Spektrum imerzivních technologií \cite{BeyondAR}}
    \label{fig:spectrum}
\end{figure}
\pagebreak
\subsection{Klíčové aspekty}
Pro správné fungování rozšířené reality jsou zásadní tyto klíčové aspekty: Jedním z nich je snímání reality, při kterém kamery a~senzory zachycují reálné prostředí a~detekují klíčové body a~plochy v~prostoru. Tento proces umožňuje správné zobrazení a~interakci digitálních objektů s fyzickým světem, čímž se vytváří dojem jejich skutečné přítomnosti v~daném prostředí. Dalším důležitým prvkem je pozicování, které zajišťuje přesné určení polohy zařízení v~prostoru. Rozšířená realita využívá různé senzory, jako jsou GPS, akcelerometry či gyroskopy, díky nimž dokáže určit orientaci a~pohyb zařízení. Tato kombinace senzorických údajů je klíčová pro správné umístění a~zobrazení digitálních prvků v~reálném světě. Poslední zásadní složkou je vykreslování, které umožňuje realistické zobrazení digitálních objektů v~souladu s perspektivou a~měřítkem okolního prostředí. Díky tomu se digitální objekty zobrazují v~přesné poloze a~působí dojmem, že jsou přirozenou součástí reality. \cite{AROverview}

\subsection{Typy rozšířené reality}
\subsubsection{Rozšířená realita pomocí markerů (marker-based)}
Tento přístup je také často nazýván rozpoznáváním obrázků, protože využívá markery umístěné v~reálném světě jako referenční body pro určení polohy a~orientace zařízení. Kamery a~senzory v~zařízení detekují marker, který slouží k identifikaci a~zobrazení specifického digitálního objektu přímo na pozici markeru. Tato metoda je jednoduchá a~efektivní pro aplikace, které vyžadují interakci s konkrétními objekty v~reálném světě. \cite{AROverview}

Markerem může být prakticky jakýkoliv grafický prvek, který je jednoznačně rozpoznatelný a~odlišný od okolí. Například speciální symbol, logo, QR kód nebo běžný obrázek s dostatečně výraznými rysy.\cite{ITJIm}

Jako příklad slouží obrázek \ref{fig:marker}, kde markerem je H na přistávací ploše a~proto se logo položí přesně na pozici markeru.

\begin{figure}[H]
    \centering
    \includegraphics[width=\textwidth]{images/marker}
    \caption{Ukázka markeru v~praxi. \cite{ITJIm}}
    \label{fig:marker}
\end{figure}

\subsubsection{Rozšířená realita bez markerů (markerless)}
Aplikace bez markerů využívá senzory, jako jsou GPS, akcelerometry a~gyroskopy pro určení polohy a~orientace zařízení v~prostoru bez použití vizuálních markerů. Tento přístup je běžně používán v~aplikacích, které závisí na geolokaci (například navigace a~mapování) a~umožňuje uživatelům interagovat s virtuálními objekty na základě jejich skutečné polohy v~reálném světě. \cite{AROverview}

Příkladem již existující markerless aplikace je \textit{IKEA Place}, která umožňuje uživatelům vizualizovat nábytek přímo v~jejich domácnosti pomocí rozšířené reality.

Na obrázku \ref{fig:ikea} lze vidět ukázku, kde si uživatel vybral produkt na e-shopu a~následně ho vizualizoval v~samotné aplikaci. \cite{IkeaPlace}

\begin{figure}[H]
    \centering
    \includegraphics[scale=0.23]{images/ikea}
    \caption{Ukázka aplikace bez markerů \textit{IKEA Place} \cite{IkeaPlace}}
    \label{fig:ikea}
\end{figure}

\subsubsection{Rozšířená realita na základě projekce}
Tento přístup využívá projekci světla na povrchy v~reálném světě k interakci s digitálními prvky. Senzory detekují změny v~projekci a~umožňují uživatelům reagovat na změny, což vytváří interaktivní zážitky. Tento typ pozicování je mnohem náročnější na výpočetní výkon, protože vyžaduje analýzu změn v~projekcích a~jejich korelaci s pohyby uživatele. \cite{AROverview}

Reálným příkladem rozšířené reality na základě projekce je interaktivní pískoviště, které se nachází ve vědecko-zábavním centru v~Liberci. Toto pískoviště poskytuje uživatelům možnost modelovat vlastní krajinu z písku a~pozorovat, jak v~ní probíhají různé přírodní procesy.

\pagebreak
Na obrázku \ref{fig:iqlandia} lze vidět pobřeží, které představuje vyvýšené části pískoviště a~vytvořené nížiny jsou zatopeny oceánem \cite{IQLANDIAGeo}
\begin{figure}[H]
    \centering
    \includegraphics[scale=0.12]{images/IQLANDIA}
    \caption{Ukázka interaktivního pískoviště IQLANDIA \cite{IQLANDIAGeo}}
    \label{fig:iqlandia}
\end{figure}
\subsubsection{Rozšířená realita pomocí překrytí}
Tento přístup umožňuje úplné nebo částečné překrytí reálného objektu digitálními informacemi. U tohoto typu je klíčové rozpoznání objektů, protože aplikace musí správně identifikovat, kde se objekt nachází, aby ho mohla správně překrýt. \cite{AROverview}

Příkladem překrytí je \textit{Google Lens}, již existující aplikace která poskytuje nástroje~pro počítání příkladů, překlad textů a~identifikaci předmětů prostřednictvím rozšířené reality. Pro lepší představu je obraz z kamery zpracován a~následně dochází k překrytí reálného textu v~cizím jazyce jeho překladem. \cite{googlelens}
\begin{figure}[H]
    \centering
    \includegraphics[scale=0.55]{images/lens.png}
    \caption{Překlad pomocí aplikace \textit{Google Lens} \cite{googlelens}}
    \label{fig:lens}
\end{figure}

\section{Přehled existujících řešení}
S rychlým rozvojem technologií rozšířené reality (AR) roste i počet aplikací, které umožňují zobrazovat digitální modely staveb přímo v~reálném prostředí uživatele. Existují však nejen aplikace určené pro prezentaci hotových či plánovaných stavebních projektů, ale také sofistikovaná řešení, která stavitelé používají přímo na staveništi. Tyto aplikace slouží například k vizualizaci rozvodů kanalizace, elektroinstalací a~vody. To přispívá k efektivnější koordinaci a~kontrole stavebních prací.
\subsection{Augment}
Dalším zajímavým řešením je aplikace Augment, která se zaměřuje na jednoduchou vizualizaci 3D modelů podobně jako prototyp vyvíjený v~rámci této práce. Na rozdíl od sofistikovaných nástrojů umožňuje snadno a~rychle zobrazit modely bez složitého nastavování. Aplikace pouze snímá povrch podlahy a~na něj položí uživatelem vybraný 3D model. \cite{augment}

Uživatelé mohou libovolně upravovat velikost, přesouvat a~otáčet objekty. Tento přístup však snižuje přesnost vztahu mezi digitálním dvojčetem a~fyzickým prostředím, a~proto není ideální pro přesnou vizualizaci stavebních projektů. Spíše se hodí na zobrazení produktů pro lepší představu například při nakupování na e-shopu. \cite{augment}

Na obrázku \ref{fig:augment} je zobrazený model obchodního stojanu v~rozšířené realitě. Uživatel může interagovat s modelem, otáčet ho a~měnit jeho velikost přímo na obrazovce.
\begin{figure}[H]
    \centering
    \includegraphics[scale=0.73]{images/augment.png}
    \caption{Ukázka pokládání modelu v~aplikaci Augment. \cite{augment}}
    \label{fig:augment}
\end{figure}

\subsection{Trimble Connect AR}
Trimble Connect AR je pokročilá aplikace využívající rozšířenou realitu pro vizualizaci stavebních projektů přímo na staveništi. Tato platforma umožňuje uživatelům zobrazovat 3D modely staveb  v~reálném prostředí přes mobilní zařízení nebo AR brýle. Na rozdíl od jednoduchých AR vizualizací se Trimble Connect AR zaměřuje na podporu profesionálů ve stavebnictví. \cite{trimble}

Mezi hlavní funkce aplikace patří možnost přesného umístění digitálního modelu na skutečné místo výstavby, synchronizace s cloudovým repozitářem projektové dokumentace a~spolupráce více uživatelů v~reálném čase. Díky integraci s BIM (Building Information Modeling) umožňuje také zobrazovat detailní informace o konstrukčních prvcích, stavebních materiálech a~dalších parametrech přímo v~AR prostředí. \cite{trimble}

Tato aplikace je ukázkou moderního využití rozšířené reality v~praxi, která přesahuje pouhou vizualizaci a~přináší přidanou hodnotu při řízení stavebních projektů. I když je funkčně mnohem pokročilejší než prototyp vyvíjený v~rámci této práce, slouží jako inspirace a~důkaz potenciálu AR v~oblasti digitálních dvojčat staveb. \cite{trimble}

Obrázek \ref{fig:trimble} znázorňuje interiér rozestavěné budovy propojený s 3D modelem technických instalací. Barevně odlišené vrstvy zobrazují jednotlivé systémy technického zařízení budovy, například zelené prvky představují vzduchotechnické vedení, oranžové částí označují nosnou konstrukci a~podobně.
\begin{figure}[H]
    \centering
    \includegraphics[scale=0.35]{images/Trimble.png}
    \caption{Ukázka zobrazení stavebního projektu v~aplikaci Trimble Connect AR. \cite{trimble}}
    \label{fig:trimble}
\end{figure}

\pagebreak
\subsection{Dalux}
Dalux nabízí několik nástrojů pro práci s BIM (Building Information Model) modely, z nichž nejrozšířenějším je Dalux BIM viewer. Tento nástroj umožňuje přehledné zobrazování stavebních projektů ve 2D a~3D formě, prohlížení výkresů, provádění měření, vytváření řezů, filtrování prvků a~zobrazování jejich vlastností. \cite{dalux}

Pro práci v~rozšířené realitě nabízí Dalux funkci TwinBIM, která je součástí mobilní aplikace Dalux. TwinBIM umožňuje umisťovat stavební modely do skutečného prostředí pomocí rozšířené reality a~zároveň podporuje interakce s objekty, například zobrazení popisu po kliknutí na objekt.\cite{daluxtwin}

Na obrázku \ref{fig:dalux} se nachází ukázka vizualizace v~Dalux TwinBIM, na které lze vidět rozmístění nábytku konstrukce budovy, reálná konstrukce je přitom zvýrazněna zelenou barvou.
\begin{figure}[H]
    \centering
    \includegraphics[scale=0.8]{images/dalux.png}
    \caption{Ukázka zobrazení stavebního projektu pomocí modulu Dalux TwinBIM \cite{daluxtwin}}
    \label{fig:dalux}
\end{figure}

\subsection{Shrnutí}
Analýza existujících řešení ukázala, že technologie pro vizualizaci stavebních projektů v~rozšířené realitě jsou již dostupné, avšak většina z nich je zaměřena na profesionální použití, je technicky složitá, nebo finančně náročná. Současná komerční řešení se často orientují na komplexní projektové řížení, simulace stavebního procesu nebo interní firemní komunikaci.

Z vypracovaného přehledu tedy vyplývá, že na trhu chybí jednoduchá, dostupná a~uživatelsky přívětivá aplikace, která by umožnila široké veřejnosti nahlížet na plánované projekty formou digitálních dvojčat v~rozšířené realitě a~zároveň poskytovat zpětnou vazbu. Tato skutečnost potvrzuje relevanci a~praktický význam této práce.