\chapter{Tabulky}

Tabulky jsou klíčovým prvkem vědeckých a technických textů. Správné použití tabulek v \LaTeX u vyžaduje pochopení dvou základních prostředí: \texttt{tabular} a \texttt{table}. Tato kapitola vás provede jejich syntaktickými rozdíly a ukázkami, jak efektivně formátovat tabulky.

\section{Prostředí \texttt{tabular}}

Prostředí \texttt{tabular} se používá k definování struktury tabulky. Umožňuje specifikovat zarovnání sloupců a přidávat vodorovné či svislé čáry.

\subsection{Syntaxe}

\begin{itemize}
    \item Parametry sloupců:
    \begin{itemize}
        \item \texttt{c} – sloupec zarovnaný na střed,
        \item \texttt{r} – sloupec zarovnaný doprava,
        \item \texttt{l} – sloupec zarovnaný doleva,
        \item \texttt{|} – svislá čára oddělující sloupce.
    \end{itemize}

    \item Oddělení buněk a řádků:
    \begin{itemize}
        \item \texttt{\&} – oddělení jednotlivých buněk v řádku,
        \item \texttt{\textbackslash\textbackslash} – označení konce řádku,
        \item \texttt{\textbackslash hline} – vložení vodorovné čáry.
    \end{itemize}
\end{itemize}

\subsection{Příklad}

\begin{center}
\begin{tabular}{|c|c|}
    \hline
    \textbf{Záhlaví 1} & \textbf{Záhlaví 2} \\
    \hline
    Řádek 1 & Hodnota 1 \\
    \hline
    Řádek 2 & Hodnota 2 \\
    \hline
\end{tabular}
\end{center}

Tento příklad vytvoří tabulku se dvěma sloupci, zarovnanými na střed, s oddělujícími svislými a vodorovnými čarami.

\section{Prostředí \texttt{table}}

Prostředí \texttt{table} umožňuje pracovat s tabulkami jako s plovoucími objekty, což znamená, že \LaTeX\ může umístit tabulku tam, kde to považuje za vhodné. Poskytuje také možnosti přidání popisků a referencí.

\subsection{Syntaxe a Příklad}

\begin{table}[h!]
    \centering
    \begin{tabular}{|c|c|}
        \hline
        \textbf{Záhlaví 1} & \textbf{Záhlaví 2} \\
        \hline
        Řádek 1 & Hodnota 1 \\
        \hline
        Řádek 2 & Hodnota 2 \\
        \hline
    \end{tabular}
    \caption{Jednoduchá tabulka s popiskem}
    \label{tab:simple_example}
\end{table}

Popisek tabulky je uveden pod tabulkou a pomocí příkazu \texttt{\textbackslash label} lze na tabulku odkazovat v textu, například: \textit{viz Tabulka \ref{tab:simple_example}}.

\section{Formátování Buněk}

Formátování tabulek v \LaTeX u zahrnuje sloučení buněk, definici vodorovných a svislých čar a nastavení šířek sloupců.

\subsection{Sloučení buněk v řádku}

Při slučování buněk v řádku se používá příkaz \texttt{\textbackslash multicolumn} (nevyžaduje balíček \texttt{multicol}!):

\begin{table}[h!]
    \centering
    \begin{tabular}{|c|c|c|}
        \hline
        \textbf{Záhlaví 1} & \textbf{Záhlaví 2} & \textbf{Záhlaví 3} \\
        \hline
        \multicolumn{3}{|c|}{Sloučené buňky} \\
        \hline
        Hodnota 1 & Hodnota 2 & Hodnota 3 \\
        \hline
    \end{tabular}
    \caption{Ukázka sloučení buněk v řádku}
    \label{tab:merged_row}
\end{table}

\newpage

\subsection{Sloučení buněk ve sloupci}

Pro sloučení buněk ve sloupci se používá příkaz \texttt{\textbackslash multirow} (vyžaduje balíček \texttt{multirow}):

\begin{table}[h!]
    \centering
    \begin{tabular}{|c|c|c|}
        \hline
        \textbf{Záhlaví 1} & \textbf{Záhlaví 2} & \textbf{Záhlaví 3} \\
        \hline
        \multirow{2}{*}{Sloučené} & Hodnota 1 & Hodnota 2 \\
        & Hodnota 3 & Hodnota 4 \\
        \hline
    \end{tabular}
    \caption{Ukázka sloučení buněk ve sloupci}
    \label{tab:merged_column}
\end{table}

\subsection{Kombinace sloučení buněk}

Můžete také kombinovat sloučení buněk v řádku i sloupci:

\begin{table}[h!]
    \centering
    \begin{tabular}{|c|c|c|}
        \hline
        \textbf{Záhlaví 1} & \textbf{Záhlaví 2} & \textbf{Záhlaví 3} \\
        \hline
        \multicolumn{2}{|c|}{\multirow{2}{*}{Kombinace}} & Hodnota 1 \\
        \cline{3-3}
        \multicolumn{2}{|c|}{} & Hodnota 2 \\
        \hline
        Hodnota 3 & Hodnota 4 & Hodnota 5 \\
        \hline
    \end{tabular}
    \caption{Ukázka kombinovaného sloučení}
    \label{tab:combined_merge}
\end{table}