\chapter{Externí přílohy\label{sec:ep}}

Externí přílohy této bakalářské práce jsou umístěny na adrese:\\ \url{https://github.com/Jiri-Fiser/thesis_ki_ujep}.

Na úložiští GitHub mohou byt uloženy tyto externí přílohy:

\begin{itemize}
\item \textbf{zdrojové kódy}
\item \textbf{doplňkové texty} (například jak instalovat aplikaci, manuály aplikace)
\item \textbf{schémata} (především, pokud se nevejdou na stranu A4 a jejich vytištění je tak problematické)
\item \textbf{screenshoty} (v textu práce lze použít jen omezený počet snímků obrazovky, které navíc nemusí být při černobílém tisku příliš přehledné)
\item \textbf{videa} (například ovládání aplikace)
\end{itemize}

V každém případě by to však měli být pouze materiály, které jste vytvořili sami. Materiály jiných autorů uvádějte v seznamu použité literatury (včetně případných odkazů na jejich originální umístění).

V této kapitole stačí uvést pouze základní strukturu úložiště (co se kde nalézá a jakou má funkci) například v podobě tabulky. 

\begin{longtable}{ll}
\hline
ki-thesis.pdf & text práce v PDF \\
ki-thesis.tex & zdrojový kód práce v \LaTeX{}u \\
kitheses.cls & definice třídy dokumentů (rozšířená třída \texttt{scrbook} \\
thesis.bib & bibliografická databáze (exportována z citace.com) \\
LOGO\_PRF\_CZ\_RGB\_standard.jpg  & logo fakulty s českým textem \\
LOGO\_PRF\_EM\_RGB\_standard.jpg  & logo fakulty s anglickým textem  \\
\hline
\end{longtable}

Všechny tyto soubory jsou potřeba pro překlad dokumentu (logo stačí jedno v příslušné jazykové verzi).

\chapter{Další přílohy}

Výjimečně může práce obsahovat i další tištěné přílohy. Obecně však dávejte přednost elektronickým přílohám umístěným na GitHubu (tato kapitola tak bude úplně chybět). 