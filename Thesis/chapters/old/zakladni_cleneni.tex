\chapter{Základní členění závěrečné práce}

\section{Kapitoly}

Závěrečná práce by měla být členěna na tři až šest kapitol nejvyšší úrovně (nepočítaje úvod a závěr). Ty by měly odpovídat osnově práce, ale nikoliv doslovně. Jiné mohou být názvy a dokonce i počet kapitol. To jest, některé body osnovy lze rozpracovat ve více kapitolách a naopak některé body lze združit do jediné kapitoly.

Důležité je, aby bylo členění přibližně rovnoměrné. Problematické jsou především kapitoly malého rozsahu (1-3 stránky) respektrive naopak kapitola zahrnující podstatnou část textu.

Kapitoly nejvyšší úrovně vždy začínají na nové stránce, která musí být lichá tj. na pravé straně. Pokud předchozí kapitola končí na liché straně, vkládá se tzv. vakát -- zcela prázdná strana. 

\section{Sekce resp. podkapitoly}

\LaTeX podporuje čtyři úrovně vnoření kapitol (kromě kapitol i sekce, podsekce, a podpodsekce), což by mělo být více než dostatečné. Druhou úrovní jsou sekce, jejichž rozsah by měl být alespoň jednu stranu, mohou však být
i výrazně delší. Zobrazují se v obsahu a jsou číslované a lze je tudíž snadno odkazovat. Měli by být použity b každé kapitole kromě úvodu a závěru.

\subsection{Podsekce}

Podsekce se už používají méně často a běžně se vyskytují i dále nečleněné sekce. Nejsou číslovány a nejsou zobrazeny v obsahu. Rozsah by měl být alespoň dva odstavce.

\subsubsection{Podpodsekce}

Podpodsekce se používají jen velmi zřídka, typicky jen u kapitol či sekcí popisující systémy s výraznou hierarchií. Tvořeny mohou být i jediným odstavcem (nikoliv však ale jedinou řádku, zde zvažte spíš použití definičního výčtu).

\section{Rozsah závěrečné práce}

Optimální rozsah \ifthenelse{\boolean{bc}}{bakalářské}{diplomové} práce není možné stanovit, neboť závisí na jejím charakteru. Práce jejíž součástí je vytvoření aplikace mohou být menšího rozsahu než práce čistě popisné. I v tomto případě by však měla mít alespoň \ifthenelse{\boolean{bc}}{třicet}{padesát⎄} stran (nepočítaje úvodní stránky až po obsah a stránky od seznamu použitých zdrojů dále a prázdné stránky tzv. vakáty).

Maximální rozsah práce je ještě obtížnější stanovit. Obecně však doporučuji v případě, že pokud rozsah (opět bez povinných částí) přesáhne osmdesát stran je vhodné uvažovat o redukci (ne všechno, co jste napsali musí být nakonec uvedeno ve finální verzi) respektive přesunu do externích příloh umístěnách na Githubu (viz příloha \vref{sec:ep}). Týká se to především různých schémat, obrazového materiálu apod.