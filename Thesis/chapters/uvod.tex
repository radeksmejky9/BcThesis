\chapter{Úvod}
Zapojení veřejnosti do územního plánování je důležitou součástí rozvoje měst. Veřejnost má mít možnost znát plánované změny ve svém okolí a vyjádřit k nim své stanovisko. V praxi však často nemají dostatečnou představu o skutečném dopadu plánovaných stavebních projektů, protože tradiční způsoby prezentace formou technických výkresů, půdorysů nebo statických vizualizací nejsou pro laickou veřejnost dostatečně srozumitelné a neumožňují pochopit prostorové vztahy mezi plánovanou stavbou a okolním prostředím.

Technologie rozšířené reality v kombinaci s konceptem digitálních dvojčat nabízí možnost výrazně zlepšit komunikaci mezi projektanty a veřejností. Rozšířená realita umožňuje zobrazit virtuální 3D model stavby přímo v reálném prostředí pomocí mobilního zařízení, což poskytuje intuitivní a snadno pochopitelnou představu o budoucím stavu. Model digitálního dvojčete přitom může obsahovat nejen vizuální podobu stavby, ale i doprovodné informace o projektu, jeho stavu a~plánovaném průběhu realizace.

Cílem této bakalářské práce je vyvinout funkční prototyp mobilní aplikace pro zobrazení digitálních dvojčat stavebních projektů v rozšířené realitě. Aplikace má umožnit vizualizaci 3D modelů staveb, poskytovat informace o projektech a umožnit uživatelům zasílat zpětnou vazbu formou hodnocení a komentářů. Součástí řešení jsou také serverová část a webové rozhraní pro správu digitálních dvojčat.

Práce je rozdělena do pěti hlavních kapitol. Teoretická část se zaměřuje na koncept digitálních dvojčat a jejich využití v územním plánování, technologii rozšířené reality a přehled existujících řešení v této oblasti. Praktická část popisuje metodiku vývoje, návrh architektury systému, volbu technologického zásobníku a implementaci systému. Kapitola diskuse vyhodnocuje výsledky implementace. Závěr shrnuje dosažené výsledky a nastiňuje možnosti budoucího rozvoje.