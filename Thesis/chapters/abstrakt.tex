\textsc{\nazevcz}

\textbf{Abstrakt:}

Tato bakalářská práce se zabývá vývojem prototypu mobilní aplikace určené pro zobrazení digitálních dvojčat stavebních projektů v prostředí rozšířené reality. V rámci řešení byl navržen~a~implementován systém založený na architektuře klient–server, který zahrnuje mobilní aplikaci, serverové REST API~a~webové rozhraní pro správu digitálních dvojčat~a~souvisejících dat. Aplikace umožňuje vyhledávání projektů prostřednictvím vlastního mapového systému, interaktivní zobrazení 3D modelů v rozšířené realitě~a~odesílání zpětné vazby ve formě komentářů~a~hodnocení. Součástí práce je také návrh~a~odvození vztahu pro transformaci GPS souřadnic do lokálního souřadnicového systému~a~implementace cache mechanismu, který optimalizuje načítání mapových dlaždic~a~zlepšuje výkon aplikace. Výsledkem je funkční prototyp, který demonstruje možnosti využití digitálních dvojčat v oblasti územního plánování~a~komunikace stavebních záměrů s~veřejností.

\textbf{Klíčová slova:} územní plánování, digitální dvojčata, mobilní aplikace, rozšířená realita
\bigskip


\textsc{\nazeven}

\textbf{Abstract:}
This bachelor’s thesis focuses on the development of~a~prototype mobile application designed for displaying digital twins of construction projects in an augmented reality environment. As part of the solution,~a~client–server system was designed and implemented, including~a~mobile application,~a~server-side REST API, and~a~web interface for managing digital twins and related data. The application enables project search through~a~custom map system, interactive 3D model visualization in augmented reality, and feedback submission in the form of comments and ratings. The work also includes the design and derivation of~a~relationship for transforming GPS coordinates into~a~local coordinate system, as well as the implementation of~a~cache mechanism to optimize map tile loading and improve application performance. The result is~a~prototype that demonstrates the potential use of digital twins in urban planning and in communicating construction projects to the public.

\textbf{Keywords:} urban planning, digital twins, mobile application, augmented reality